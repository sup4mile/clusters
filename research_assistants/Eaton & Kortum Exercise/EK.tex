\documentclass[11pt, assign]{assign}
\usepackage{lipsum, tikz}

\title{3-Country, 1-Sector Version of Eaton \& Kortum Model of International Trade}
\instructions{Solve this simple version of the Eaton \& Kortum model numerically using the Julia programming language.}

\begin{document}
%[Course number] \hfill    [Instructor]
%
%[Course title] \hfill [Academic term] \\

\hrule
\vspace{3mm}
\maketitle
\hrule


% document class options:
% - draft: show both problems and solutions -> use this to typeset the assignment
% - assign: show only problems
% - solution: show only solutions
% - grading: show solutions and grading instructions
% - showtotal: adds up points and displays point total for each problem
% - showsubtotal: adds up points and displays point total for each subproblem
% - neither of these options: runs through the problem set three times: typesetting assign, solution, and grading.

% with documentclass option qbank you can filter problems based on labels, difficulty, and quality:
% \filter[and]{Nash equilibrium, continuum of actions}, \filter[or]{Nash equilibrium, Bayesian Nash equilibrium} filters labels
% \difficulty{concept>=3} \difficulty{math<=3}, where difficulty={concept, math}
% \quality{instructive=2}, \quality{fun=3}, where quality={instructive, fun}

%\begin{problem}[title=Problem Title] %, difficulty={3, 1}, quality={2, 4}]
%
%\lipsum[1][1-5]
%
%\hint{Your students may need some hint.}
%
%\answer This starts your answer. You can tell your TA how many points each argument is worth\pt[2].
%\end{problem}


\begin{problem}[title=Simple Version of Ricardian Trade Model,source={Eaton and Kortum, 2002}]% , labels={Bayesian Nash equilibrium, continuum of actions}] %, difficulty={1, 3}, quality={5, 2}]

Consider a three-country world, where the endowment of labor in each country is 1. There is a continuum of goods $k$ and the technology for producing a good $k$ in country $i$ is $y_{i}(k)=z_{i}(k) \ell_{i}(k)$. Preferences are of constant elasticity of substitution form, with the elasticity set to 2. Assume that the distributions over $z$ are Fr\'echet with $\theta=4$.

\begin{subproblems}

\question Simple and symmetric. Let there be no trade costs, i.e., $\tau_{n i}=0$, and let $T_{i}=1.5$ for all $i$. Report the equilibrium bilateral trade share matrix. (An element of the matrix is $\pi_{n i}$, the share of total spending in $n$ on goods from i.) The solution to this model is trivial, so this is a good place to first check that our programs are working. \label{q1}

\question Symmetric geography. Now introduce iceberg trade costs. Let $\tau_{n i}=0.1$ for each $n \neq i$, and keep the remaining parameters as in part \ref{q1} Report the bilateral trade share matrix.

\question Asymmetric geography. Countries have identical technologies, $T_{i}=1.5$ for all $i$, and $\theta=4$. Country 3 , however, is "far away" from countries 1 and $2: \tau_{12}=\tau_{21}=1.05$ and $\tau_{13}=\tau_{31}=\tau_{32}=\tau_{23}=1.3$. Report the equilibrium bilateral trade share matrix. (An element of the matrix is $\pi_{n i}$, the share of total spending in $n$ on goods from $i$.) Report an index of welfare in each country, $w_{i} / P_{i}$, where $P_{i}$ is the CES aggregate price index. [My equilibrium wage vector is: $(1,1,0.966)$.]

\question Technological progress. Let $T_{2}=3$, and keep the remaining parameters as in part (c). This is a technological improvement in country 2. [You will need to redraw your productivities.] Report the bilateral trade share matrix. Compare welfare in this economy with welfare in the economy in part (c). Discuss your findings in the context of how an increase in technology in one country benefits other countries. [My equilibrium wage vector is: $(1,1.14,0.962)$.]

\question Fr\'echet dispersion. Now change $\theta=8$, and keep the remaining parameters as in part (c) (i.e., change $T_{2}=1.5$ ). [You will need to redraw your productivities.] Report the bilateral trade share matrix. Compare welfare in this economy with welfare in the economy in part (c). What is the intuition for this result? How does the change in $\theta$ affect countries 1 and 2 compared to country 3? Why? [My equilibrium wage vector is: (1, 1, 0.978).]

\end{subproblems}

\end{problem}

\end{document}